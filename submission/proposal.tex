\documentclass[11pt]{article}
\usepackage{amsmath,amsthm,amssymb}
\usepackage{graphicx}
\usepackage{fontspec}
\setmainfont{Arial}
\usepackage{wrapfig}
\usepackage[export]{adjustbox}
\usepackage[toc,page]{appendix}
\usepackage[font=footnotesize]{caption}
\usepackage[margin=0.5in]{geometry}
\usepackage[singlespacing]{setspace}
\usepackage[small, compact]{titlesec}  % http://ctan.mirrors.hoobly.com/macros/latex/contrib/titlesec/titlesec.pdf
\titleformat*{\subsubsection}{\slshape}
\usepackage{paralist}
%\renewenvironment{thebibliography}[1]{
%    \footnotesize
%    \let\par\relax\let\newblock\relax
%    \inparaenum[\bfseries{[}1{]}]}{\endinparaenum}
\usepackage[english]{babel}
\pagenumbering{arabic}
\usepackage{lineno}
%\linenumbers % remove before submitting
\usepackage{hyperref}
\hypersetup{colorlinks=true, urlcolor=blue, linkcolor=black, citecolor=black}

\begin{document} % 7 page limit
\sloppy
\begin{center}
\large{\textbf{
    Functional Activity of the Human Gut Microbiome to Classify Colorectal Cancer
}} \\
\vspace{11pt}
\small{
    Kelly L. Sovacool \\
    August 2020
}
\end{center}

\input{submission/aims.tex}

\section*{Background and Motivation} % 1/2 - 1 page

genetics and environmental factors explain only a small proportion of disease incidence, so we turn to the microbiome \cite{thomas_metagenomic_2019}.

other studies found specifical potential functional pathways that are CRC biomarkers, especially choline metabolism pathway \cite{thomas_metagenomic_2019}.


we don't know enough about functional redundancy in the human gut microbiome \cite{heintz-buschart_human_2018}

``Community-level function is often more conserved than community composition15,39–41, con- sistent with a functional repertoire ‘defining’ a niche and satisfied by different microbial assemblages.'' \cite{franzosa_species-level_2018}

many published studies claiming to have found functional redundancies in microbial systems lack quantitative analyses of redundancy \cite{souza_metagenomic_2015, ferrer_microbiota_2013}.

no one agrees on exact definition of functional redundancy \cite{louca_function_2018, heintz-buschart_human_2018, tully_dynamic_2018, royalty_quantitative_2020}.
``Stability in ecosystem function with increasing microbial diversity is often considered an empirical indication of functional redundancy''
\cite{royalty_quantitative_2020}


trait contribution evenness (TCE): ``the evenness in relative contribution of that trait among taxa within the community... This definition has several appealing properties including: TCE is an extension of established diversity theory, functional redundancy measurements from communities with different richness and relative trait contribution by taxa are easily comparable, and any quantifiable trait data (genes copies, protein abundance, transcript copies, respiration rates, etc.) is suitable for analysis.'' \cite{royalty_quantitative_2020}

``Functional redundancy is a measure of the number of different populations within a community that are able to perform the same functions. Functional redundancy can increase functional resilience, in case perturbations affect the taxonomic community structure; this allows for a return to community function, and therefore can increase stability.'' \cite{heintz-buschart_human_2018}

``This functional redundancy is further reflected in the fraction of the observed microbial community capable of participating in each metabolic step, with no statistically significant difference between the boreholes, except for ammonia oxidation (Figure 6).'' (used Student's t-test and Wilcoxon rank sum) \cite{tully_dynamic_2018}

\section*{Significance} % 1/2 page


\section*{Research Design and Methods}

\subsection*{Aim 1. Functional redundancy of the gut microbiome}

\subsubsection*{1A) Build profiles of taxonomic composition and functional gene potential.}

16S rRNA gene sequencing was previously performed on stool samples from patients in the GLNE 007 cohort for classification modeling to detect CRC \cite{baxter_microbiota-based_2016}.
Since then, additional samples have been collected and sequenced, bringing the total dataset to 211 CRC and 223 non-cancerous samples.
Sequences will be processed with mothur according to the MiSeq SOP \cite{schloss_introducing_2009, kozich_development_2013}.
Briefly, processing steps include filtering for quality, removing chimeric sequences, clustering sequences into OTUs using the \textit{de novo} OptiClust method with a similarity threshold of 97\%,
and generating a table of OTU abundances by samples \cite{westcott_opticlust_2017}.
Abundances will be rarefied and converted to relative abundances to circumvent biases in sampling depth across samples.
This final OTU abundance table will serve as the taxonomic composition profiles of each community.

Whole metagenome shotgun sequencing will be performed and metagenomes will be processed with HUMAnN2 \cite{franzosa_species-level_2018} to characterize functional potential of the CRC and non-cancerous microbial communities.
Sequences will be filtered and trimmed for quality prior to processing with HUMAnN2.
HUMAnN2 uses MetaPhlan2 to screen sequences against a curated reference of 400,000 clade-specific marker genes to detect the microbial species present in each sample \cite{segata_metagenomic_2012}.
This strategy is assembly-free and saves considerable computational resources over assembly-based methods.
Next, sequences are mapped to annotated reference genomes to identify the gene families defined by Uniref90 and the metabolic pathways defined by Metacyc \cite{capsi_metacyc_2017} that are encoded by each community.
MinPath pares down the list of metabolic pathways to the minimum set that can be explained by the genes encoded in each metagenome \cite{ye_parsimony_2009}.
The end result is a conservative table of metabolic pathways encoded by each microbial community and their abundances.
As with OTU abundances, pathway abundances will be converted to relative abundances.
This table of pathway abundances will serve as the functional potential profiles of each community.


\subsubsection*{1B) Determine presence and degree of functional redundancy in CRC and non-cancerous gut microbiomes.}

How humann2 paper assessed fcnl potential w/in and b/e communities:
Calculate diversity metrics on function abundances to find "contributional diversity".
Alpha (within-sample): Gini-Simpson. Beta (between-sample): Bray-Curtis.
Fig 2A in \cite{franzosa_species-level_2018} plotted beta vs alpha for samples.

``A function contributed by a single species has low within-sample (‘simple’) contributional diversity, while a function with many equal contributors has high within-sample (‘complex’) contributional diversity. If a function is contributed by the same assemblage of species across samples, it has low between-sample (‘conserved’) contributional diversity, whereas a function contributed by different assemblages has high between-sample (‘variable’) contributional diversity.'' \cite{franzosa_species-level_2018}


Maybe do Wilcoxon rank-sum on Gini-Simpson, and ANOSIM + NMDS on Bray-Curtis distances with a post hoc multivariate Tukey test. \cite{hannigan_diagnostic_2018}

\subsubsection*{1C) Build and compare CRC classification models with taxonomic composition and functional gene potential.}

random forest or logistic regression

\subsection*{Aim 2. Integrating active metabolites with functional gene potential}

\subsubsection*{2A) Annotate compounds from untargeted mass spectrometry and select known products of bacterial metabolism.}

GNPS

inspired by AMON, but use MetaCyc instead of KEGG (MetaCyc has more pathways)

\subsubsection*{2B) Calculate the intersection of pathways from active metabolites and functional potential profiles.}

\subsubsection*{2C) Build and compare CRC classification models with all potential pathways or only confirmed active pathways.}

random forest or logistic regression

\section*{Potential Outcomes and Conclusions}

limitation: genes with unknown functions

limitation: mass-spec features with unknown identity

\pagebreak
\bibliographystyle{submission/nihunsrt} % downloaded from https://www.latextemplates.com/template/nih-grant-proposal
%\section*{\refname}
\footnotesize{
\bibliography{submission/prelim.bib}
\par}
\end{document}
