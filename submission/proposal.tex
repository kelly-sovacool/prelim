\documentclass[11pt]{article}
\usepackage{amsmath,amsthm,amssymb}
\usepackage{graphicx}
\usepackage{fontspec}
\setmainfont{Arial}
\usepackage{wrapfig}
\usepackage[toc,page]{appendix}
\usepackage[font=footnotesize]{caption}
\usepackage[margin=0.5in]{geometry}
\usepackage[singlespacing]{setspace}
\usepackage[small, compact]{titlesec}  % http://ctan.mirrors.hoobly.com/macros/latex/contrib/titlesec/titlesec.pdf
\titleformat*{\subsubsection}{\slshape}
\usepackage{paralist}
\renewenvironment{thebibliography}[1]{
    \let\par\relax\let\newblock\relax
    \inparaenum[\bfseries{[}1{]}]}{\endinparaenum}
\usepackage[english]{babel}
\pagenumbering{gobble}
\usepackage{lineno}
%\linenumbers % remove before submitting

\begin{document} % 7 page limit
\sloppy
\begin{center}
\large{\textbf{
    Functional Activity of the Human Gut Microbiome to Classify Colorectal Cancer
}} \\
\vspace{11pt}
\small{
    Kelly L. Sovacool \\
    August 2020
}
\end{center}

\section*{Specific Aims} % 1 page

Changes in the taxonomic composition and metabolic activity of human microbiomes have been observed in several diseases.
In the case of colorectal cancer (CRC), evidence of toxigenic activity by gut microbes implies that these changes are not only a response to disease, but may also play a role in disease etiology.
Taxonomic composition is commonly defined by amplicon sequencing of the 16S rRNA gene and clustering sequences into Operational Taxonomic Units (OTUs).
Previous studies have built OTU-based machine learning models to classify stool samples as normal or cancerous, to serve as a less invasive diagnostic tool for CRC than colonoscopy.
Efforts to find consistent changes in taxonomic composition of microbiomes between normal and dysbiotic states have found mixed success, in part because interpersonal variability in taxonomic composition sometimes exceeds the variability between disease states.
Variability of microbiome composition between individuals with the same disease status may be explained by functional redundancy, where different microbial species carry out the same functions and thus can replace each other with little effect on the overall function of the community.

Sequencing whole metagenomes to identify the genes present and annotate known gene functions is commonly used to build a profile of functional potential of the microbiome.
Combining taxonomic composition from OTUs with functional potential from metagenomes allows one to characterize functional redundancy across communities, where communities with similar functional potential have different taxonomic composition.
Untargeted mass spectrometry can validate the functional potential characterized from metagenomics by identifying metabolites that are active in a community, thus painting a more precise picture of active microbial functions.
Here, I propose to investigate the impacts of taking functional redundancy and active metabolites into account on human stool sample classification for CRC diagnosis.

\subsection*{Aim 1. Assess the impact of functional redundancy of the gut microbiome on CRC classification.}
\textit{Hypothesis: Using functional gene profiles instead of only taxonomic profiles improves the classification modeling of samples as CRC or non-cancerous because of functional redundancy in the gut microbiome.}
\begin{compactenum}[A.]
    \item Build taxonomic profiles with OTUs from 16S rRNA gene sequences and build profiles of functional gene potential from metagenomes.
    \item Compare taxonomic composition to functional gene potential of microbiomes within and between disease states to determine presence and degree of functional redundancy.
    \item Build machine learning models to classify samples as CRC or non-cancerous with taxonomic composition, functional gene potential profiles, or both as model features and compare performance.
\end{compactenum}

\subsection*{Aim 2. Assess the impact of integrating active metabolites with functional gene potential on CRC classification.}
\textit{Hypothesis: Using active metabolic pathways confirmed with mass spectrometry instead of all potential metabolic pathways from metagenomes improves the classification modeling of samples as CRC or non-cancerous.}
\begin{compactenum}[A.]
    \item Annotate compounds from untargeted mass spectrometry with the GNPS database and select those known to be products of bacterial metabolic pathways with the MetaCyc database.
    \item Calculate the intersection of pathways associated with active metabolites and the pathways from functional potential profiles from metagenomes.
    \item Build machine learning models to classify samples as CRC or non-cancerous with all potential metabolic pathways or only confirmed active metabolic pathways as model features and compare performance.
\end{compactenum}

\section*{Dataset}
Stool samples were collected from patients undergoing colonoscopy as part of the GLNE 007 study (https://clinicaltrials.gov/ct2/show/study/NCT00843375).
211 individuals were diagnosed with CRC and 223 were confirmed non-cancerous.
16S rRNA gene amplicon sequencing was performed and remaining stool was kept frozen.
Part of the remaining stool will be used for whole metagenome shotgun sequencing and untargeted tandem mass spectrometry to complete these aims.

\section*{Background and Motivation} % 1/2 - 1 page

genetics and environmental factors explain only a small proportion of disease incidence, so we turn to the microbiome \cite{thomas_metagenomic_2019}.

``it is not possible to classify individuals as having healthy colons or screen relevant neoplasia using Bray- Curtis distances based on the 16S rRNA gene sequences collected from fecal samples (see Fig. S1 in the supplemental material). This variation is likely due to the ability of many bacterial populations to fill the same niche such that different populations cause the same disease in different individuals. Furthermore, a growing number of studies have shown that it is rare for a single bacterial species to be associated with a disease. Instead, subsets of the microbiome account for differences in health.'' \cite{topcuoglu_framework_2020}

other studies found specifical potential functional pathways that are CRC biomarkers, especially choline metabolism pathway \cite{thomas_metagenomic_2019}.


we don't know enough about functional redundancy in the human gut microbiome \cite{heintz-buschart_human_2018}

``Community-level function is often more conserved than community composition15,39–41, con- sistent with a functional repertoire ‘defining’ a niche and satisfied by different microbial assemblages.'' \cite{franzosa_species-level_2018}

many published studies claiming to have found functional redundancies in microbial systems lack quantitative analyses of redundancy \cite{souza_metagenomic_2015, ferrer_microbiota_2013}.

no one agrees on exact definition of functional redundancy \cite{louca_function_2018, heintz-buschart_human_2018, tully_dynamic_2018, royalty_quantitative_2020}.
``Stability in ecosystem function with increasing microbial diversity is often considered an empirical indication of functional redundancy''
\cite{royalty_quantitative_2020}


trait contribution evenness (TCE): ``the evenness in relative contribution of that trait among taxa within the community... This definition has several appealing properties including: TCE is an extension of established diversity theory, functional redundancy measurements from communities with different richness and relative trait contribution by taxa are easily comparable, and any quantifiable trait data (genes copies, protein abundance, transcript copies, respiration rates, etc.) is suitable for analysis.'' \cite{royalty_quantitative_2020}

``Functional redundancy is a measure of the number of different populations within a community that are able to perform the same functions. Functional redundancy can increase functional resilience, in case perturbations affect the taxonomic community structure; this allows for a return to community function, and therefore can increase stability.'' \cite{heintz-buschart_human_2018}

``This functional redundancy is further reflected in the fraction of the observed microbial community capable of participating in each metabolic step, with no statistically significant difference between the boreholes, except for ammonia oxidation (Figure 6).'' (used Student's t-test and Wilcoxon rank sum) \cite{tully_dynamic_2018}

\section*{Significance} % 1/2 page


\section*{Research Design and Methods}

\subsection*{Aim 1. Functional redundancy of the gut microbiome}

\subsubsection*{1A) Build profiles of taxonomic composition and functional gene potential.}

16S rRNA gene sequencing was previously performed on stool samples from patients in the GLNE 007 cohort for classification modeling to detect CRC \cite{baxter_microbiota-based_2016}.
Since then, additional samples have been collected and sequenced, bringing the total dataset to 211 CRC and 223 non-cancerous samples.
Sequences will be processed with mothur according to the MiSeq SOP \cite{schloss_introducing_2009, kozich_development_2013}.
Briefly, processing steps include filtering for quality, removing chimeric sequences, clustering sequences into OTUs using the \textit{de novo} OptiClust method with a similarity threshold of 97\%,
and generating a table of OTU abundances by samples \cite{westcott_opticlust_2017}.
Abundances will be rarefied and converted to relative abundances to circumvent biases in sampling depth across samples.
This final OTU abundance table will serve as the taxonomic composition profiles of each community.

Whole metagenome shotgun sequencing will be performed and metagenomes will be processed with HUMAnN2 \cite{franzosa_species-level_2018} to characterize functional potential of the CRC and non-cancerous microbial communities.
Sequences will be filtered and trimmed for quality prior to processing with HUMAnN2.
HUMAnN2 uses MetaPhlan2 to screen sequences against a curated reference of 400,000 clade-specific marker genes to detect the microbial species present in each sample \cite{segata_metagenomic_2012}.
This strategy is assembly-free and saves considerable computational resources over assembly-based methods.
Next, sequences are mapped to annotated reference genomes to identify the gene families defined by Uniref90 and the metabolic pathways defined by Metacyc \cite{capsi_metacyc_2017} that are encoded by each community.
MinPath pares down the list of metabolic pathways to the minimum set that can be explained by the genes encoded in each metagenome \cite{ye_parsimony_2009}.
The end result is a conservative table of metabolic pathways encoded by each microbial community and their abundances.
As with OTU abundances, pathway abundances will be converted to relative abundances.
This table of pathway abundances will serve as the functional potential profiles of each community.


\subsubsection*{1B) Determine presence and degree of functional redundancy in CRC and non-cancerous gut microbiomes.}

How humann2 paper assessed fcnl potential w/in and b/e communities:
Calculate diversity metrics on function abundances to find "contributional diversity".
Alpha (within-sample): Gini-Simpson. Beta (between-sample): Bray-Curtis.
Fig 2A in \cite{franzosa_species-level_2018} plotted beta vs alpha for samples.

``A function contributed by a single species has low within-sample (‘simple’) contributional diversity, while a function with many equal contributors has high within-sample (‘complex’) contributional diversity. If a function is contributed by the same assemblage of species across samples, it has low between-sample (‘conserved’) contributional diversity, whereas a function contributed by different assemblages has high between-sample (‘variable’) contributional diversity.'' \cite{franzosa_species-level_2018}


Maybe do Wilcoxon rank-sum on Gini-Simpson, and ANOSIM + NMDS on Bray-Curtis distances with a post hoc multivariate Tukey test. \cite{hannigan_diagnostic_2018}

\subsubsection*{1C) Build and compare CRC classification models with taxonomic composition and functional gene potential.}

Random forest models will be built to classify samples as CRC or non-cancerous using OTU abundances, metabolic pathway abundances, or both as model features.
The random forest method has been found to perform well for microbiome-based classification problems because it can be used for non-linear data and accounts for interactions between features \cite{baxter_microbiota-based_2016}.
The dataset will be randomly split into 80\% training and 20\% testing sets, stratified to maintain the proportion of CRC to non-cancerous samples.

\subsection*{Aim 2. Integrating active metabolites with functional gene potential}

\subsubsection*{2A) Annotate compounds from untargeted mass spectrometry and select known products of bacterial metabolism.}

GNPS

inspired by AMON, but use MetaCyc instead of KEGG (MetaCyc has more pathways)

\subsubsection*{2B) Calculate the intersection of pathways from active metabolites and functional potential profiles.}

\subsubsection*{2C) Build and compare CRC classification models with all potential pathways or only confirmed active pathways.}

random forest or logistic regression

\section*{Potential Outcomes and Conclusions}

limitation: genes with unknown functions

limitation: mass-spec features with unknown identity

\pagebreak
\bibliographystyle{submission/nihunsrt} % downloaded from https://www.latextemplates.com/template/nih-grant-proposal
%\section*{\refname}
\footnotesize{
\bibliography{submission/prelim.bib}
\par}
\end{document}
